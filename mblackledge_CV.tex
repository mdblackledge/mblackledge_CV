% !TEX TS?program = pdflatexmk
\documentclass[]{mbcv}

\usepackage{xcolor}
\usepackage{marvosym}
\usepackage{array}
\usepackage{hyperref}
\pagenumbering{gobble}
\usepackage{afterpage}
\usepackage{fontawesome5}
\definecolor{background}{HTML}{e1e1e1} 
\definecolor{border}{HTML}{e1e1e1} 
\usepackage[backend = biber, sorting=ydnt, maxbibnames=99]{biblatex}
\newcolumntype{P}[1]{>{\centering\arraybackslash}p{#1}}
%\renewcommand{\refname}{Publications}

\begin{document}

\namesection{Matthew David Blackledge}{PhD, MSc, BSc}

\noindent\fcolorbox{border}{background}{%
\begin{minipage}[t][0.85\textheight]{0.3\textwidth}
\section{Personal Details}
\vspace{5pt}
 \Writinghand\hspace*{3pt}  17 Elmhurst Lodge\\
\hspace*{12pt} Christchurch Park\\
\hspace*{12pt} Sutton, SM2 5TY\\
\hspace*{12pt} United Kingdom

\vspace*{5pt}

\Letter\hspace*{3pt}   \href{matthew.blackledge@icr.ac.uk}{matthew.blackledge@icr.ac.uk}

\vspace*{5pt}

\Telefon\hspace*{3pt}  +44 (0)7581 016 932

\vspace*{5pt}

\Mundus\hspace*{6pt}
\href{https://sites.google.com/view/blackledgelab}{blackledgeimaging.co.uk}

\vspace*{5pt}

\faLinkedin\hspace*{4pt} \href{https://www.linkedin.com/in/mdblackledge/}{linkedin.com/in/mdblackledge}

\vspace*{5pt}

\faGithub\hspace*{3pt} \href{https://github.com/mdblackledge}{mdblackledge}

\vspace*{5pt}

\sectionsep
\section{Education}
\vspace{5pt}
\descript{Institute of \newline Cancer Research}
\location{PhD in Medical Physics}
2007-2011
\sectionsep

\descript{Surrey University}
\location{MSc in Medical Physics}
2006-2007 (Distiction)
\sectionsep

\descript{Imperial College}
\location{BSc in Physics}
2003-2006

\vspace*{5pt}

\sectionsep
\section{Programming}
\vspace{5pt}
\location{Languages}
 C \textbullet{}  C++ \textbullet{} Objective-C  \\
Python \textbullet{} Python-C API   \\
 \LaTeX \textbullet{} R \textbullet{} HTML \textbullet{} IDL\\
 gRPC \textbullet{} Matlab\\
 
\vspace{5pt}
\location{Administrative Tools}
Unix \textbullet{} Conda \textbullet{} Jupyter  \\
Git \textbullet{} Confluence \textbullet{} Jira \\

\vspace{5pt}
\location{Machine Learning}
TensorFlow \textbullet{} Keras \textbullet{} PyTorch  \\
Scikit-Learn \textbullet{} SimpleITK

\vspace*{5pt}
\sectionsep
\section{Languages}
English \textbullet{} Polish

\end{minipage}}%
\hfill
\begin{minipage}[t]{0.65\textwidth}

\section{Professional Experience}
\descript{Institute of Cancer Research}

\location{Team Leader of Computational Imaging | 2019-present}
I lead a team of 6 PhD students, 3 postdoctoral researchers and one software engineer.

Here is some more text

\vspace*{10pt}
\begin{tightemize}

\item Leading on the development of \textbf{whole-body MRI radiomics} and advanced \textbf{statistical inference} approaches for assessing inter-lesion response heterogeneity in patients receiving systemic therapies for advanced prostate cancer.

\end{tightemize}

\sectionsep

\location{NIHR Postdoctoral Fellow | 2012-2019}
\vspace*{10pt}

\begin{tightemize}

\item Leading on the development of \textbf{whole-body MRI radiomics} and advanced \textbf{statistical inference} approaches for assessing inter-lesion response heterogeneity in patients receiving systemic therapies for advanced prostate cancer.

\item Innovating \textbf{machine-learning} approaches to identify and monitor intra-lesion response heterogeneity in soft-tissue sarcoma treated with radiotherapy using multi-parametric imaging.

\item Developing approaches to improve the accuracy and repeatability of quantitative MRI through new \textbf{model-fitting} paradigms.

\item Pioneering new \textbf{in-silico} approaches for validation of imaging biomarkers using advanced \textbf{histological analyses} of preclinical cancer models.

\item Steering and managing the development of new \textbf{regulatory approved software} for automatic response assessment of patients with metastatic cancer using whole-body MRI.

\item Developed novel \textbf{image segmentation} approaches for whole-body MRI and PET/CT in patients with advanced prostate cancer and lymphoma.

\item Invented and implemented a new \textbf{software} platform (pyOsiriX) that accelerates prototyping of imaging research tools, and facilitates the interaction between imaging scientists and radiologists through a familiar image viewing platform.

\item Led whole-body \textbf{MRI protocol development} and promoted its use within multi-center clinical trials. 

\end{tightemize}

\sectionsep

\location{Postdoctoral Researcher | 2011-2012}

\begin{tightemize}

\item Investigated \textbf{multi-modality imaging} (PET/MRI) for monitoring the response of Lymphoma to chemotherapy.

\item Developed spatial \textbf{registration strategies} for sequential MRI and PET studies.

\end{tightemize}

\sectionsep

\end{minipage}

\newpage
\afterpage{%
\section{Awards and Prizes}
\vspace*{10pt}
\begin{tightemize}
\item First prize for presentation to the MR of Cancer Study Group at the annual meeting of the International Society for Magnetic Resonance in Medicine (ISMRM) in 2017. 
\item Sylvia Lawler prize from the Royal Society of Medicine in 2013 for development of innovative imaging methodologies for quantification response heterogeneity from whole-body MRI.
\item Two summa cum laude awards at the ISMRM annual meetings in 2013 and 2016.
\item Certificate of merit award at the annual meeting of the European Society for Magnetic Resonance in Medicine and Biology (ESMRMB) in 2008.
\end{tightemize}
\sectionsep

\section{Leadership and Mentoring}
\vspace*{10pt}
\begin{tightemize}

\item Recruitment and management of staff involved in the NIHR I4I award, including oversight of software development and ensuring final regulatory approval.

\item Management of grant funds and resources, both within personal fellowships and project awards.

\item Regular delivery of timely updates to funding bodies highlighting the progress of my research; authorship of final reports and lay summaries to research ethics committees to convey clinical trial results. 

\item Chair regular meetings to ensure that research project milestones are reached on time.

\item Co-supervision of two PhD students: providing mentorship on (i) the statistical modelling and image analysis of noisy MRI data, and (ii) machine-learning approaches for advanced analysis of histopathology. 

\item Hosted and supervised a summer-student through funding awarded by the Institute of Cancer Research (ICR) to investigate novel methodologies for atlas-based segmentation of multi-modal whole-body imaging data (MRI and CT). 

\item Mentored a one-year placement for an international postgraduate student exploring spatial registration methods between PET-CT and MRI.

\end{tightemize}
\sectionsep

\section{Scientific Outreach}
\vspace*{10pt}
\begin{tightemize}
\item Invited key speaker at prominent international and national scientific meetings.
\item Principle lecturer at the annual mathematics and imaging course at the Royal Marsden Hospital (RMH) and ICR (interactive slides: \href{https://github.com/mattblackledge/mathematicsofimaging}{https://github.com/mattblackledge/mathematicsofimaging}).
\item Run and deliver an annual course on the use of LaTeX for graduate students within the ICR. 
\item Invited member of the scientific editorial board for European Radiology.
\item Regular peer-reviewer for a number of relevant journals (including European Radiology and Magnetic Resonance in Medicine).
\item Invited as principle lecturer for a course on the physics of MRI at the Centre for Advanced Studies in Warsaw, Poland.
\item Co-written book chapters on the on the implementation and optimisation of whole-body MRI in oncology.
\item Acted as a STEM (Science, Technology, Engineering and Mathematics) ambassador to promote careers in science.
\end{tightemize}
\sectionsep


\section{Research Collaborations}
\vspace*{10pt}
\begin{tightemize}
\item Continually collaborate with clinical staff at the RMH to develop protocols for clinical trials.
\item Develop optimised MRI protocols at institutions across the UK to support multi-centre clinical imaging trials.
\item Spearheading a collaboration with University College London (UCL) to investigate the use of deep learning for automatic MRI image quality control.
\item Member of the European Diffusion Weighted Imaging initiative in Myeloma (EDWIM), acting as leading MR-physicist to develop imaging recommendations that promote standardization of acquisition and reporting of WB-MRI in myeloma.
\item Established a collaboration with Norwegian University of Science and Technology (NTNU) to investigate the use of advanced histopathological analysis for MRI biomarker validation.
\item Jointly established a collaboration with an industrial partner (Mint Medical GmbH) to support software development to a commercial standard.
\end{tightemize}

\newpage

\section{Grant Funding}
\begin{table}[htp]
\begin{center}
\begin{tabular}{|P{0.15\linewidth}|P{0.5\linewidth}|P{0.1\linewidth}|P{0.15\linewidth}|}
\hline
\textbf{Name of funding organisation} & \textbf{Project Title} & \textbf{Amount awarded} & \textbf{Role} \\
\hline
National Institute of Health Research (NIHR) & Real-world testing of software for measuring bone disease on whole-body MRI in patients with prostate cancer and myeloma	& £1,955,730 & Co-Investigator \\
\hline
NIHR Biomedical Research Centre & PhD studentship: AI for improved patient outcome prediction in brain metastasis &	£149,770	& Principle Investigator\\
\hline
MedTech SuperConnector & Next generation whole-body imaging for painful metastatic disease & £88,740 & Principle Investigator \\
\hline
CRUK Centre for Convergence Science & PhD studentship: Combined functional MRI and super-resolution ultrasound for non-invasive monitoring of tumour blood delivery during radiotherapy of breast cancer & £149,770 & Co-Principal Investigator \\
\hline
CRUK Centre for Convergence Science & Application of novel 3D super-resolution ultrasound imaging for radiotherapy response assessment in breast tumours & £29,954 & Co-Principal Investigator \\
\hline
Global Challenges Research Fund & CCR5115 KORTUC phase II - Randomised phase II trial testing efficacy of intra-tumoural hydrogen peroxide as a radiation sensitiser in patients with locally advanced/recurrent breast cancer & £ 79,930 & Co-investigator \\
\hline
Cancer Research UK & Improving neoadjuvant therapy in high-risk sarcoma & £1,493,584 & Co-Investigator\\
\hline
Sarcoma UK & Using MRI for response assessment of soft-tissue sarcoma to pre-operative radiotherapy & £24,490 & Principle Investigator \\
\hline
Cancer Research UK & Predicting the location of lung nodule occurrence from low-dose CT using convolutional time-to-event networks (X-Net) & £99,470 & Principle Investigator \\
\hline
Institute of Cancer Research & Career Development Faculty start-up package & £950,000 & Principle Investigator \\
\hline
NIHR & Advanced computer diagnostics for whole body magnetic resonance imaging to improve management of patients with metastatic bone cancer & £1,201,674 & Co-Investigator \\
\hline
NIHR & Evaluating treatment response of whole-body metastases with multi-modality imaging & £242,628 & Principle Investigator \\
\hline
\end{tabular}
\end{center}
\end{table}


\begin{refsection}[papers.bib]
\nocite{*}
\section{Publications}
\vspace*{10pt}
\printbibliography[heading=none]
\end{refsection}
\begin{refsection}[patents.bib]
\nocite{*}
\section{Patents}
\vspace*{10pt}
\printbibliography[heading=none]
\end{refsection}
\begin{refsection}[abstracts.bib]
\nocite{*}
\section{International Conferences}
\vspace*{10pt}
\begin{flushleft}
I have contributed to 44 peer-reviewed articles accepted for presentation at international conferences.  Those for which I was a first or senior author are listed below.
\end{flushleft}
\printbibliography[heading=none]
\end{refsection}
\clearpage
\restoregeometry
}
%\bibliographystyle{habbrvyr}
%\bibliography{papers}
%
%\bibliography{abstracts}

\end{document}