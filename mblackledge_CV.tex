% !TEX TS?program = pdflatexmk
\documentclass[]{mbcv}

\usepackage{xcolor}
\usepackage{marvosym}
\usepackage{array}
\usepackage{hyperref}
\pagenumbering{gobble}
\usepackage{afterpage}
\usepackage{ragged2e}
\usepackage{fontawesome5}
\definecolor{background}{HTML}{e1e1e1} 
\definecolor{border}{HTML}{e1e1e1} 
\usepackage[backend = biber, sorting=ydnt, maxbibnames=99]{biblatex}
\newcolumntype{P}[1]{>{\centering\arraybackslash}p{#1}}
%\renewcommand{\refname}{Publications}

\newcommand{\detailsvspace}{8pt}

\begin{document}

\namesection{Matthew David Blackledge}{PhD, MSc, BSc}

\noindent\fcolorbox{border}{background}{%
\begin{minipage}[t][0.9\textheight]{0.3\textwidth}
\section{Personal Details}
\vspace{\detailsvspace}
 \Writinghand\hspace*{3pt}  17 Elmhurst Lodge\\
\hspace*{12pt} Christchurch Park\\
\hspace*{12pt} Sutton, SM2 5TY\\
\hspace*{12pt} United Kingdom

\vspace*{5pt}

\Letter\hspace*{3pt}   \href{matthew.blackledge@icr.ac.uk}{matthew.blackledge@icr.ac.uk}

\vspace*{5pt}

\Telefon\hspace*{3pt}  +44 (0)7581 016 932

\vspace*{5pt}

\Mundus\hspace*{6pt}
\href{https://sites.google.com/view/blackledgelab}{blackledgeimaging.co.uk}

\vspace*{5pt}

\faLinkedin\hspace*{4pt} \href{https://www.linkedin.com/in/mdblackledge/}{linkedin.com/in/mdblackledge}

\vspace*{5pt}

\faGithub\hspace*{3pt} \href{https://github.com/mdblackledge}{mdblackledge}

\vspace*{\detailsvspace}

\sectionsep
\section{Education}
\vspace{\detailsvspace}
\descript{Institute of \newline Cancer Research}
\location{PhD in Medical Physics}
2007-2011
\sectionsep

\descript{Surrey University}
\location{MSc in Medical Physics}
2006-2007 (Distiction)
\sectionsep

\descript{Imperial College}
\location{BSc in Physics}
2003-2006

\vspace*{\detailsvspace}

\sectionsep
\section{Programming}
\vspace{\detailsvspace}
\location{Languages}
 C \textbullet{}  C++ \textbullet{} Objective-C  \\
Python \textbullet{} Python-C API   \\
 \LaTeX \textbullet{} R \textbullet{} HTML \textbullet{} IDL\\
 gRPC \textbullet{} Markdown  \\
 
\vspace{5pt}
\location{Administrative Tools}
Unix \textbullet{} Conda \textbullet{} Jupyter  \\
Git \textbullet{} Confluence \textbullet{} Jira \\

\vspace{5pt}
\location{Machine Learning}
TensorFlow \textbullet{} Keras \textbullet{} PyTorch  \\
Scikit-Learn \textbullet{} SimpleITK

\vspace*{\detailsvspace}
\sectionsep
\section{Languages}
English \textbullet{} Polish

\vspace*{30pt}

\end{minipage}}%
\hfill
\begin{minipage}[t]{0.65\textwidth}

\section{Professional Experience}
\location{Director | 2021-present, Diafora Medical Ltd.}
Co-founded Diafora Medical Ltd. to commercialise my Artificial Intelligence algorithm for accelerating acquisition of clinical Diffusion-Weighted MRI scans. Responsibilities include active engagement in board meetings, overall strategy of company direction, authorship of patents, leading clinical trial development, and raising more than £3 million to begin development of the invented techniques.

\vspace{10pt}

\location{Team Leader of Computational Imaging | 2019-present \newline Institute of Cancer Research}
Lead a team of postdoctoral fellows, and PhD/MSc students, combining quantitative imaging (MRI and CT) with innovative image processing and artificial intelligence to solve clinical problems in oncology. Focus on real-world solutions for prostate, breast, head and neck, gynaecological, and connective tissue cancers, with core ambitions to:
 
\vspace*{20pt}
\begin{tightemize}
\item Promote personalised risk stratification with \textbf{deep-learning} and \textbf{radiomics} by discovering new imaging biomarkers of response and prognosis.
\item Reduce clinician burden using \textbf{domain-adaptation AI} for automatic segmentation of tumours and organs at risk during radiotherapy planning.
\item Establish new biomarkers of heterogeneous tumour response through \textbf{bayesian and mathematical modelling} of quantitative MRI.
\item Improve the patient experience and departmental capacity of quantitative MRI by innovating AI-accelerated \textbf{image reconstruction}.
\item Decipher the link between tumour biology and quantitative MRI using \textbf{computational histopathology} and super-resolution ultrasound.
\end{tightemize}

\sectionsep

\location{Postdoctoral Research Fellow | 2011-2019, Institute of Cancer Research}
Won multiple rounds of competitive funding to develop automated segmentation of total body disease from whole-body MRI, and turn the technology into an approved medical device. Achievements include:

\vspace*{10pt}
\begin{tightemize}
\item Leading whole-body MRI protocol development within the department and promoting its use within multi-center clinical trials. 
\item Developing novel AI-based image segmentation approaches for whole-body MRI and PET/CT in patients with advanced prostate cancer and lymphoma.
\item Steering development of regulatory approved software for automatic assessment of patients with metastatic cancer using whole-body MRI.
\item Pioneering in-silico approaches to correlate quantitative MRI with biology in preclinical prostate cancer models using histological analysis.
\item Innovating open-source software (pyOsiriX) to accelerate prototyping of medical imaging research tools.

\end{tightemize}

\sectionsep

\location{Consultancy Positions}
\textbf{Nomela (2021-present):} Expert advisory panel member and core AI data scientist developing clinical   teleradiology for diagnosis of cancerous moles.
\newline
\newline
\textbf{Bayer AG (2019-present):} Lead MRI protocol development for multi-center testing of novel pharmaceuticals, along with bespoke image analysis software.


\end{minipage}

\newpage
\section{Leadership and Mentoring}
\vspace*{10pt}
\begin{tightemize}
\item Experience of leading small-medium research teams in image-processing, MRI physics, and artificial intelligence; currently six PhD students and three postdoctoral fellows (one PhD student have recently defended their thesis).
\item Mentored seven summer studentships/interns in the development of innovative image processing and/or MRI development projects.
\item Management of grant funds and resources, both within personal fellowships and project awards.
\item Regular delivery of timely updates to funding bodies highlighting the progress of my research; authorship of final reports and lay summaries to research ethics committees to convey clinical trial results. 
\item Chair regular meetings to ensure that research project milestones are reached on time.
\item Lead a number of clinical trials, including development of overall research methodology, statistical analysis plans, power calculations, imaging protocol manuals, and patient information sheets. 

\end{tightemize}
\sectionsep

\section{Scientific Outreach and Teaching}
\vspace*{10pt}
\begin{tightemize}
\item Invited key speaker at prominent international and national scientific meetings.
\item Principle lecturer at the annual mathematics and imaging course at the Royal Marsden Hospital (RMH) and ICR (interactive slides: \href{https://github.com/mattblackledge/mathematicsofimaging}{https://github.com/mattblackledge/mathematicsofimaging}).
\item Run and deliver courses on the use of LaTeX for graduate students within the ICR. 
\item Invited member of the scientific editorial board for European Radiology.
\item Regular peer-reviewer for a number of relevant journals (including European Radiology, Magnetic Resonance in Medicine, and Computers in Biology and Medicine).
\item External examiner of the digital health MSc course at St Andrews University.
\item Deliver and annual lecture series on Diffusion Weighted MRI and Imaging Biomarkers at the Higher Specialist Scientist Training programme (University of Manchester).
\item Invited member of the ESMRMB congress planning committee in 2020 and 2021.
\end{tightemize}
\sectionsep


\section{Research Collaborations}
\vspace*{10pt}
\begin{tightemize}
\item Continually collaborate with clinical staff at the RMH to develop protocols for clinical trials.
\item Develop optimised MRI protocols at institutions across the UK to support multi-centre clinical imaging trials.
\item Member of the European Diffusion Weighted Imaging initiative in Myeloma (EDWIM), acting as leading MR-physicist to develop imaging recommendations that promote standardisation of acquisition and reporting of WB-MRI in myeloma.
\item Established a collaboration with Norwegian University of Science and Technology (NTNU) to investigate the use of advanced histopathological analysis for MRI biomarker validation.
\item Jointly established a collaboration with an industrial partner (Mint Medical GmbH) to support software development to a commercial standard.
\end{tightemize}

\section{Awards and Prizes}
\vspace*{10pt}
\begin{tightemize}
\item Commended for Innovation Award, Institute of Cancer Research (2022)
\item First prize for presentation to the MR of Cancer Study Group at the annual meeting of the International Society for Magnetic Resonance in Medicine (2017). 
\item Sylvia Lawler prize from the Royal Society of Medicine for development of innovative imaging methodologies for quantification response heterogeneity from whole-body MRI (2013).
\item Summa cum laude awards at the ISMRM annual meetings (2013 and 2016).
\item Certificate of merit award at the annual meeting of the European Society for Magnetic Resonance in Medicine and Biology (2008).
\end{tightemize}

\newpage

\section{Grant Funding}
\begin{table}[htp]
\begin{center}
\begin{tabular}{|P{0.15\linewidth}|P{0.5\linewidth}|P{0.1\linewidth}|P{0.15\linewidth}|}
\hline
\textbf{Name of funding organisation} & \textbf{Project Title} & \textbf{Amount awarded} & \textbf{Role} \\
\hline
National Institute of Health Research (NIHR) & Real-world testing of software for measuring bone disease on whole-body MRI in patients with prostate cancer and myeloma	& £1,955,730 & Co-Investigator \\
\hline
NIHR Biomedical Research Centre & PhD studentship: AI for improved patient outcome prediction in brain metastasis &	£149,770	& Principle Investigator\\
\hline
MedTech SuperConnector & Next generation whole-body imaging for painful metastatic disease & £88,740 & Principle Investigator \\
\hline
CRUK Centre for Convergence Science & PhD studentship: Combined functional MRI and super-resolution ultrasound for non-invasive monitoring of tumour blood delivery during radiotherapy of breast cancer & £149,770 & Co-Principal Investigator \\
\hline
CRUK Centre for Convergence Science & Application of novel 3D super-resolution ultrasound imaging for radiotherapy response assessment in breast tumours & £29,954 & Co-Principal Investigator \\
\hline
Global Challenges Research Fund & CCR5115 KORTUC phase II - Randomised phase II trial testing efficacy of intra-tumoural hydrogen peroxide as a radiation sensitiser in patients with locally advanced/recurrent breast cancer & £ 79,930 & Co-investigator \\
\hline
Cancer Research UK & Improving neoadjuvant therapy in high-risk sarcoma & £1,493,584 & Co-Investigator\\
\hline
Sarcoma UK & Using MRI for response assessment of soft-tissue sarcoma to pre-operative radiotherapy & £24,490 & Principle Investigator \\
\hline
Cancer Research UK & Predicting the location of lung nodule occurrence from low-dose CT using convolutional time-to-event networks (X-Net) & £99,470 & Principle Investigator \\
\hline
Institute of Cancer Research & Career Development Faculty start-up package & £950,000 & Principle Investigator \\
\hline
NIHR & Advanced computer diagnostics for whole body magnetic resonance imaging to improve management of patients with metastatic bone cancer & £1,201,674 & Co-Investigator \\
\hline
NIHR & Evaluating treatment response of whole-body metastases with multi-modality imaging & £242,628 & Principle Investigator \\
\hline
\end{tabular}
\end{center}
\end{table}


\begin{refsection}[papers.bib]
\nocite{*}
\section{Publications}
\vspace*{10pt}
\printbibliography[heading=none]
\end{refsection}
\begin{refsection}[patents.bib]
\nocite{*}
\section{Patents}
\vspace*{10pt}
\printbibliography[heading=none]
\end{refsection}
\begin{refsection}[abstracts.bib]
\nocite{*}
\section{International Conferences}
\vspace*{10pt}
\begin{flushleft}
I have contributed to well over 70 peer-reviewed articles accepted for presentation at international conferences.  Those for which I was a first or senior author are listed below.
\end{flushleft}
\printbibliography[heading=none]
\end{refsection}
\clearpage
\restoregeometry

%\bibliographystyle{habbrvyr}
%\bibliography{papers}
%
%\bibliography{abstracts}

\end{document}